% !TEX TS-program = xelatex
% !TEX encoding = UTF-8 Unicode
% !Mode:: "TeX:UTF-8"

\documentclass{resume}
\usepackage{zh_CN-Adobefonts_external} % Simplified Chinese Support using external fonts (./fonts/zh_CN-Adobe/)
%\usepackage{zh_CN-Adobefonts_internal} % Simplified Chinese Support using system fonts
\usepackage{linespacing_fix} % disable extra space before next section
\usepackage{cite}

\begin{document}
\pagenumbering{gobble} % suppress displaying page number

\name{杜伟夫  博士}

\basicInfo{
  \email{duweifu@gmail.com} \textperiodcentered\ 
  \phone{(+86) 156-186-05158} \textperiodcentered\ 
  \linkedin[weifu]{duweifu@gmail.com}}
  
\section{\faRss\  研究方向及兴趣}
\begin{itemize}[parsep=0.5ex]
  \item 自然语言处理
  \item 机器学习
  \item 情感分析
  \item 知识图谱
  \item 大数据风控
  \item 并行计算
\end{itemize}

 
\section{\faBook\  学术论文及专利}

\datedsubsection{\textbf{Expert Systems with Applications(SCI indexed, IF:2.596)}}{2010}
\textit{Optimizing Modularity to Identify the Semantic Orientation of Chinese Words
}
\datedsubsection{\textbf{WSDM}}{2010}
\textit{Adapting Information Bottleneck Method for Automatic Construction of Domain-oriented Sentiment Lexicon}

\datedsubsection{\textbf{Journal of Computer Research and Development}}{2009}
\textit{A New Method to Compute Semantic Orientation}

\datedsubsection{\textbf{NAACL-HLT}}{2009}
\textit{An Iterative Reinforcement Approach for Fine-Grained Opinion Mining}

\datedsubsection{\textbf{CIKM}}{2009}
\textit{Improved Information Bottleneck based Domain-oriented Sentiment Lexicon Construction}

\datedsubsection{\textbf{CN 102236722 B}}{2014}
\textit{一种基于三元组的用户评论摘要的生成方法与系统
}

\datedsubsection{\textbf{专利申请中}}{2015}
\textit{一种基于迁移学习的用户账号打通方法与系统
}

\section{\faUsers\ 主要学术背景及工作经历}
\begin{itemize}[parsep=0.5ex]
 \item 2014-2017年任职百度(中国)有限公司,担任算法负责人,负责十亿量级的账户打通系统的算法研发。
 \item 2010-2013年任职大众点评网,担任资深算法工程师与技术负责人,先后负责搜索排序,搜索质量,团购推荐等项目的算法研发。
  \item 2010年毕业于哈尔滨工业大学,获工学博士学位,研究方向为自然语言处理。
\end{itemize}

% Reference Test
%\datedsubsection{\textbf{Paper Title\cite{zaharia2012resilient}}}{May. 2015}
%An xxx optimized for xxx\cite{verma2015large}
%\begin{itemize}
%  \item main contribution
%\end{itemize}

\section{\faCogs\ 专业技能}
% increase linespacing [parsep=0.5ex]
\begin{itemize}[parsep=0.5ex]
  \item 编程语言: C++, Python, Haskell, Scala, R
  \item 数据平台: Hadoop, Spark
  \item 操作系统:Linux, OS X, Windows
\end{itemize}


%% Reference
%\newpage
%\bibliographystyle{IEEETran}
%\bibliography{mycite}
\end{document}
